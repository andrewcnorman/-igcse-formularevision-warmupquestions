%\documentclass[a4paper,12pt]{article}
\documentclass[14pt]{beamer}
\usepackage[latin1]{inputenc}

\title[Revision]{Warm-up problems}
\author{\href{http://andrewcnorman.net/}{\textsc{A.C. Norman}}}
\date{}

%\usepackage{beamerarticle}
\usepackage{acn-scientific}
%\usetikzlibrary{decorations.pathreplacing,
%  decorations.markings,
%  decorations.pathmorphing}
\usetikzlibrary{arrows,shapes,positioning}
\usetikzlibrary{circuits.ee.IEC}
%\usepackage{cancel}

\usepackage{isotope}
\usepackage{cancel}

\usetheme{radleykeys}

\usepackage{enumerate}

\setbeamertemplate{enumerate item}{\textbf{\arabic{enumi}}}
\setbeamertemplate{enumerate subitem}{(\alph{enumii})}
\setbeamertemplate{enumerate subsubitem}{(\roman{enumiii})}

\setlength{\leftmargini}{0.5em}
%\leftmarginii
%\leftmarginiii

\setlength{\unitlength}{1cm}

\begin{document}
%\maketitle

\begin{frame}{}
\titlepage
\vspace*{2em}
\ccbyncsa
\end{frame}

% MEANING OF QUESTION LABELS (a copy of this is at the bottom of the document)
%
% labels refer to equations
% e.g. 2.3 means 2=F=ma (see below), 3=looking for a (1=F, 2=m, 3=a)
%
% 1) v=d/t
% 2) F=ma
% 3) a=Delta v/t (1=a, 2=v, 3=u, 4=t)
% 4) \rho = m/V
% 5) W=Fd
% 6) KE=1/2 mv^2
% 7) GPE = mgh
% 8) W=mg
% 9) p=F/A
% 10) Q=It
% 11) V=IR
% 12) P=IV
% 13) E=QV
% 14) v=f\lambda
% 15) n=sin(big)/sin(small)
% 16) sin c = 1/n (1=c 2=n)
% 17) efficiency = useful energy input / total energy output
% 18) p=\rho g h
% 19) nuclear equation (1=numbers, 2=alpha, 3=beta, 4=fission, 5=fusion)
%
%%%%% TRIPLE ONLY
% T1) p=mv
% T2) M=Fd
% T3) input (primary) voltage / output (secondary) voltage = primary turns / secondary turns
%     (1=V_p, 2=V_s, 3=N_p, 4=N_s)
% T4) V_p I_p = V_s I_s (1=V_p, 2=I_p, 3=V_s, 4=I_s)

\begin{frame}{Warm up problems: don't forget ESAU!}
\begin{enumerate}
\item\label{2.3} If an object of mass \SI{5}{kg} experiences a force of \SI{3}{N}, what is its acceleration?
\item\label{3.1} A car's speed increases from \SI{25}{m/s} to \SI{60}{m/s} in \SI{5}{s}.  What is its acceleration?
\item\label{6.1} If the mass of the car is \SI{1200}{kg}, what is its final kinetic energy at \SI{60}{m/s}?
\item\label{19} How many protons, neutrons and electrons are there in a \isotope[108][]{Ag^+} ion?
\item\label{11.2} If a \SI{9}{V} battery is connected to a circuit of resistance \SI{45}{\ohm}, what current flows?
\end{enumerate}
\end{frame}

\begin{frame}{Solutions (1)}
\begin{enumerate}
\item\label{2.3} $F=ma$, $a=\dfrac{F}{m}=\dfrac{\SI{3}{N}}{\SI{5}{kg}}=\SI{0.6}{m/s^{2}}$
\item\label{3.1} $a=\dfrac{\Delta v}{t}=\dfrac{\SI{60}{m/s}-\SI{25}{m/s}}{\SI{5}{s}}=\dfrac{\SI{35}{m/s}}{\SI{5}{s}}=\SI{7}{m/s^{2}}$
\item\label{6.1} $KE=\frac{1}{2}mv^2=\frac{1}{2}\times\SI{1200}{kg}\times(\SI{60}{m/s})^2=\SI{2160000}{J}$
\item\label{19} 47 protons, 61 neutrons and 46 electrons
\item\label{11.2} $V=IR$, $I=\dfrac{V}{R}=\dfrac{\SI{9}{V}}{\SI{45}{\ohm}}=\SI{0.2}{A}$
\end{enumerate}
\end{frame}

\begin{frame}{Warm up problems (2): use ESAU!}
\begin{enumerate}
\item\label{10.2} If a charge of \SI{2}{C} flows through an ammeter in \SI{5}{s}, what will the current read?
\item\label{3.2} A car starting at a speed of \SI{5}{m/s} accelerates at \SI{3}{m/s^{2}} for \SI{12}{s}.  What is its final speed?
\item\label{10.1} If a battery supplies a current of \SI{0.8}{A} for  \SI{28}{s}, what charge has flowed?
\item\label{4.3} A pure gold ring has mass \SI{15}{g} and gold has density \SI{19.6}{g/cm^{3}}.  What is its volume?
\item\label{10.3} If a lightbulb has a current of \SI{1.2}{A} flowing through it, how long will it take for \SI{90}{C} of charge to flow through it?
\end{enumerate}
\end{frame}

\begin{frame}{Solutions (2)}
\begin{enumerate}
\item $I=\dfrac{Q}{t}=\dfrac{\SI{2}{C}}{\SI{5}{s}}=\SI{0.4}{A}$
\item $a=\dfrac{\Delta v}{t}$, $\Delta v = at = \SI{3}{m/s^{2}}\times\SI{12}{s} = \SI{36}{m/s}$\\
$\text{Starting speed} = \SI{5}{m/s}$, so $\text{final speed}=\SI{41}{m/s}.$
\item $Q=It=\SI{0.8}{s}\times\SI{28}{s}=\SI{22.4}{C}$
\item $\rho=\dfrac{m}{V}$, $V=\dfrac{m}{\rho}=\dfrac{\SI{15}{g}}{\SI{19.6}{g/cm^{3}}}=\SI{0.77}{cm^{3}}$
\item $Q=It$, $t=\dfrac{Q}{I}=\dfrac{\SI{90}{C}}{\SI{1.2}{A}}=\SI{75}{s}$
\end{enumerate}
\end{frame}

\begin{frame}{Warm up problems (3): use ESAU!}
\begin{enumerate}
\item\label{1.1} A car drives 82.0 miles in 3~h 46~min.  What is its average speed
\begin{enumerate}
\item in mph?
\item in m/s? [Hint: 1 mile = 1600 m]
\end{enumerate}
\item\label{3.1} A Nissan LEAF accelerates to \SI{27.5}{m/s} from rest in \SI{11.5}{s}.  What is its acceleration?
\item An electric car charging point supplies a voltage of \SI{394}{V} and a current of \SI{104}{A}.
\begin{enumerate}
\item\label{11.3} What is the resistance of the car charging (connected to the charging point)?
\item\label{10.1} The car is connected for \SI{29}{min}.  How much charge has flowed?
\end{enumerate}
\end{enumerate}
\end{frame}

\begin{frame}{Solutions (3)}
\begin{enumerate}
\item \begin{enumerate}
\item $v=\dfrac{d}{t}=\dfrac{\SI{82.0}{mile}}{\SI{3}{h}+\frac{46}{60}~\si{h}}=\SI{21.8}{mph}$
\item $21.8~\dfrac{\si{mile}}{\si{h}}\times\dfrac{\SI{1600}{m}}{\si{mile}}\times\dfrac{\SI{1}{h}}{\SI{60}{min}}\times\dfrac{\SI{1}{min}}{\SI{60}{s}}=\SI{9.7}{m/s}$
\end{enumerate}
\item $a=\dfrac{\Delta v}{t}=\dfrac{\SI{27.5}{m/s}}{\SI{11.5}{s}}=\SI{2.39}{m/s^{2}}$
\item \begin{enumerate}
\item $V=IR$, $R=\dfrac{V}{I}=\dfrac{\SI{394}{V}}{\SI{104}{A}}=\SI{3.79}{\ohm}$
\item $Q=It=\SI{29}{min}\times\dfrac{\SI{60}{s}}{\si{min}}\times\SI{104}{A}=\SI{181000}{C}$
\end{enumerate}
\end{enumerate}
\end{frame}

\begin{frame}{Warm up problems (4): use ESAU!}
\vspace*{-0.8em}\begin{enumerate}
\item\label{17.1} A solar panel generates \SI{300}{J} of electrical energy for every \SI{1400}{J} of light energy.  What is its efficiency?
\item\label{11.3} A flashlamp bulb operates at a voltage of \SI{2.5}{V} and a current of \SI{0.3}{A}.  What is its resistance?
\item\label{2.1} A Nissan LEAF has mass \SI{1557}{kg}.  If it accelerates at \SI{2.3}{m/s^{2}}, what is the unbalanced force on it?
\item\label{8.1} A man has mass of \SI{77.5}{kg}.  What is his weight on the Moon (gravitational field strength = \SI{1.63}{N/kg})?
\item\label{15.2} A laser beam entering water (refractive index = 1.33) refracts at \SI{12}{\degree} to the normal. What was the angle of incidence?
\end{enumerate}
\end{frame}

\begin{frame}{Solutions (4)}
\vspace*{-0.8em}\begin{enumerate}
\item\label{17.1} $\text{efficiency} = \dfrac{\text{useful energy out}}{\text{total energy in}}\times100\% = \dfrac{\SI{300}{J}}{\SI{1400}{J}}\times100\% = 21.4\%$
\item\label{11.3} $V=IR$, $R=\dfrac{V}{I}=\dfrac{\SI{2.5}{V}}{\SI{0.3}{A}}=\SI{8.33}{\ohm}$
\item\label{2.1} $F=ma = \SI{1557}{kg}\times\SI{2.3}{m/s^{2}}=\SI{3580}{N}$
\item\label{8.1} $W=mg=\SI{77.5}{kg}\times\SI{1.63}{N/kg}$
\item\label{15.2} $n=\dfrac{\sin(\text{big})}{\sin(\text{small})}$, $\sin(\text{big})=n \sin(\text{small})=1.33\times\sin(\SI{12}{\degree})=0.27652\ldots$\\
big$=\sin^{-1}(0.27652\ldots)=\SI{16.1}{\degree}$
\end{enumerate}
\end{frame}

\begin{frame}{Warm up problems (5): use ESAU!}
\vspace*{-0.8em}\begin{enumerate}
\item\label{15.3} A light beam hits glass (refractive index = 1.52) at an angle of incidence of \SI{48}{\degree}.  What is the angle of refraction?
\item\label{5.2} A man pushed a car \SI{8}{m}, doing \SI{1800}{J} of work in the process.  What force did he push the car with?
\item\label{10.2} \SI{78}{C} of charge flow through an ammeter in \SI{32.5}{s}.  What current does the ammeter read?
\item\label{6.1} A marble of mass \SI{25}{g} rolls at \SI{0.6}{m/s} along a track.  What is its kinetic energy?
\item\label{13.1} \SI{45}{C} of electrical charge leave a \SI{9}{V} battery.  How much energy does this charge carry?
\end{enumerate}
\end{frame}

\begin{frame}{Solutions (5)}
\vspace*{-0.8em}\begin{enumerate}
\item\label{15.3} $n=\dfrac{\sin(\text{big})}{\sin(\text{small})}$, $\sin(\text{small})=\dfrac{\sin(\text{big})}{n}=\dfrac{\sin(\SI{48}{\degree})}{1.52}=0.488911\ldots$\\
small=$\sin^{-1}(0.488911\ldots)=\SI{29.3}{\degree}$
\item\label{5.2} $W=Fd$, $F=\dfrac{W}{d}=\dfrac{\SI{1800}{J}}{\SI{8}{m}}=\SI{225}{N}$
\item\label{10.2} $Q=It$, $I=\dfrac{Q}{t}=\dfrac{\SI{78}{C}}{\SI{32.5}{s}}=\SI{2.4}{A}$
\item\label{6.1} $\text{KE}=\frac{1}{2}mv^{2}=\frac{1}{2}\times\SI{0.025}{kg}\times(\SI{0.6}{m/s})^2=\SI{4.5e-3}{J}$
\item\label{13.1} $E=QV=\SI{45}{C}\times\SI{9}{V}=\SI{405}{J}$
\end{enumerate}
\end{frame}

\begin{frame}{Warm up problems (X): use ESAU!}
\vspace*{-0.8em}\begin{enumerate}
\item\label{X} 
\item\label{X} 
\item\label{X} 
\item\label{X} 
\item\label{X} 
\end{enumerate}
\end{frame}

\end{document}

%A rough and ready way to get random(ish) numbers in a bash shell (GNU/Linux)
%shuf -i 1-19 -n 1
%shuf -i 1-3 -n 1

\begin{frame}{Warm up problems (X): use ESAU!}
\vspace*{-0.8em}\begin{enumerate}
\item\label{X} 
\item\label{X} 
\item\label{X} 
\item\label{X} 
\item\label{X} 
\end{enumerate}
\end{frame}

% MEANING OF QUESTION LABELS
%
% labels refer to equations
% e.g. 2.3 means 2=F=ma (see below), 3=looking for a (1=F, 2=m, 3=a)
%
% 1) v=d/t
% 2) F=ma
% 3) a=Delta v/t (1=a, 2=v, 3=u, 4=t)
% 4) \rho = m/V
% 5) W=Fd
% 6) KE=1/2 mv^2
% 7) GPE = mgh
% 8) W=mg
% 9) p=F/A
% 10) Q=It
% 11) V=IR
% 12) P=IV
% 13) E=QV
% 14) v=f\lambda
% 15) n=sin(big)/sin(small)
% 16) sin c = 1/n (1=c 2=n)
% 17) efficiency = useful energy input / total energy output
% 18) p=\rho g h
% 19) nuclear equation (1=numbers, 2=alpha, 3=beta, 4=fission, 5=fusion)
%
%%%%% TRIPLE ONLY
% T1) p=mv
% T2) M=Fd
% T3) input (primary) voltage / output (secondary) voltage = primary turns / secondary turns
%     (1=V_p, 2=V_s, 3=N_p, 4=N_s)
% T4) V_p I_p = V_s I_s (1=V_p, 2=I_p, 3=V_s, 4=I_s)
